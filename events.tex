\section{Events}
\label{sec:events}

\subsubsection{Group Meetings}
\label{sec:group_meetings}
Group meetings will be held
weekly
in the CfAO Atrium interaction
area, typically on Mondays at 10am
(Zoom link \href{https://ucsc.zoom.us/j/7207759741}{https://ucsc.zoom.us/j/7207759741}).
We discuss current events,
research activities, university
happenings, etc. Group members
may be asked to share their
current successes and challenges,
or present slides with advanced
notice.
Group meetings are mandatory
for CARG members.

\subsubsection{Cosmo Club}
\label{sec:cosmo_club}

The cosmology and galaxy
formation-focused weekly
seminar is Cosmo Club, 
which is held
Mondays 12:30pm - 1:30pm in
ISB 102. CARG members should
attend this weekly seminar
and sign up 
to meet with speakers.

\subsubsection{Astronomy Colloquia}
\label{sec:astro_colloquia}

The Astronomy colloquia are a
critical event for the
intellectual culture of the
Astronomy department, and all
members of CARG should attend
every week as possible.
They are held on Wednesdays
at 3:30-4:30 PM, usually in
Nat Sci Annex 101.
There is usually
coffee and cookies starting
at 3:00 PM in the CfAO atrium.\\

\noindent
Usually, a meeting schedule is 
created for the colloquium speaker each
week. Please sign up to meet
with every colloquium speaker as
schedule permits.\\

\noindent
The
\href{https://www.astro.ucsc.edu/news-events/Seminars/index.html}{Astronomy Colloquium Schedule is available here}.

\subsubsection{Physics Colloquia}
\label{sec:phys_colloquia}

The Physics Department Colloquia are usually
scheduled from 3:45 pm-4:55 pm on Thursdays
in Physical Sciences 114. They serve
cookies at 3:20pm in the ISB foyer. 
CARG members are encourage to attend
this optional event.\\

\noindent
The
\href{https://www.physics.ucsc.edu/news-events/colloquia/index.html}{Physics Colloquium Schedule is available here}.

\subsubsection{FLASH}
\label{sec:flash}

The weekly informal departmental
seminar is the
Friday Lunch Time Astrophysics Seminar,
which is held
Fridays 12:30pm - 1:30pm in ISB 102.
CARG members should attend this
weekly event.\\

\noindent
In the Spring, second year 
graduate students
usually present their research at
FLASH. It's especially important
to attend FLASH during this
time to support our students.
