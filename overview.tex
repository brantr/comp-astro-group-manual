\section{Overview}
\label{sec:overview}

The UCSC Computational Astrophysics Research Group (CARG) conducts research in a variety of topics related to galaxy formation and evolution. While we work primarily on theory and simulation, we also apply artificial intelligence/machine learning to astrophysics and perform analysis on large-scale datasets.


\subsubsection{Galaxy Formation and Evolution}
Galaxy and structure formation represent fundamental concepts in our cosmological models for the evolution of the universe from inflationary perturbations to the rich collection of galaxies we see today. Our research group studies the primary physical mechanisms and modalities that give rise to the observed distribution of galaxy properties. To this end, we primarily use numerical simulations and calculations in an attempt to improve the physical realism of our models.

\subsubsection{AI/Machine Learning in Astrophysics}
AI and machine learning represent some of the most exciting technologies in science, with wide applicability to astrophysical data, theory, and simulation. The Computational Astrophysics Research Group applies deep learning to analyze and model astronomical observations, understand our astrophysical simulations, and emulate difficult calculations with rapid approximations.

\subsubsection{Astrophysical Fluids and Turbulence}
Understanding the fate of the baryonic material that comprises gas and stars in and in between galaxies is a primary goal of theoretical astrophysics. These components generate the primary observable signatures of the galaxy formation process, and are therefore key for connecting our theoretical models to observational probes. Our groups uses numerical simulation to model the properties of astrophysical fluids, which are often turbulent and complex, including dense regions of the interstellar medium that give rise to star formation, the interaction between supernovae explosions and the interstellar medium, the connection between outflowing galactic-scale winds and the circumgalactic medium, and the eventual return of gas and metals to the diffuse intergalactic medium.

\subsubsection{Numerical Simulation Methodologies}
Our group pushes the boundaries of simulation methodologies used in astrophysics by exploiting computational architectures, such as NVIDIA and AMD GPUs, to perform calculations faster and with better power efficiency while maintaining physical realism and accuracy.

\subsubsection{Testing Theory with Observation}
Effective theories share close connections with observations, and theory provides guidance for conducting observing programs with a large scientific return. Our research group is heavily involved in using theory to inform observational programs with existing and upcoming facilities like Hubble Space Telescope, James Webb Space Telescope, Nancy Grace Roman Space Telescope, Large Synoptic Survey Telescope, the Atacama Large Millimeter/submillimeter Array, and many others.

\subsection{About This Manual}

This manual is intended to provide resources and clarity for members of the Computational Astrophysics Research Group. This is a living document and will be updated as needed to reflect changes in university policy, inaccuracies, typographical errors, and to better assist CARG members in their work. We reserve
the right to change this document as necessary.

Note that this manual is {\bf not intended} to override, augment,
 or supersede any labor agreement between the University
of California and organized workers, such as that defined by the \href{https://ucnet.universityofcalifornia.edu/resources/employment-policies-contracts/bargaining-units/graduate-student-researchers/contract/}{UC Graduate Student Researchers contract} or the
\href{https://ucnet.universityofcalifornia.edu/resources/employment-policies-contracts/bargaining-units/postdoctoral-scholars/contract/}{UC Postdoctoral Scholars contract}. We are happy to
adjust this document as needed to address any issues related to this point, and please contact
Brant Robertson with any such requests. In all cases, where this document and the UC contracts
could possibly disagree, the UC contracts supersede this document. This manual is not intended as a legally-binding document nor should it be interpreted as providing legal advice.
