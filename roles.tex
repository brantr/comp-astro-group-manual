\section{Roles \& Expectations}
\label{sec:roles}

This section outlines the typical work responsibilities members of the Comp Astro Research Group.
For everyone in the CARG,
the typical work schedule and mode is described in Section \ref{sec:logistics}.
The terms of employment are provided in the university offer letters, which typically come from the Dean's hiring authority.

\subsubsection{Faculty}

Brant's primary responsiblity
is to enable the research
of other CARG members. He
must keep the lights on,
find resources, edit
papers, and serve on departmental
and professional committees. These
responsibilities are in addition to his
teaching role.\\

\noindent
He also will occasionally try to
write his own papers!


\subsubsection{Postdoctoral Scholars}

Postdocs in the CARG have
a primary responsibility
to pursue their research 
projects. A postdoctoral research position is a nominal 40-hour per week 
in-person time commitment for a 100\% FTE.

\subsubsection{Graduate Students}

Graduate students in the CARG have
different responsibilities
depending on whether they
are on TA, GSR, or summer GSR, and
their career stage.

\begin{itemize}
	\item For students being supported on a TA, the role is entirely academic and focused on making nominal progress toward the PhD degree (including thesis research).
	\item For students being supported on a GSR during the academic year, the role is a combination of academic efforts making nominal progress toward the PhD and conducting directed research. The directed research portion funded by the GSR during the academic year is a nominal 20-hour per week time commitment for a 50\% FTE.
	\item For students on summer GSR, the role is completely directed research funded by the GSR. The
	directed research portion funded by the GSR during the summer quarter is a nominal 40-hour per week time commitment for a 100\% FTE.
\end{itemize}

\noindent
When working on a paid GSR, work activites that count toward the employment are pre-approved by the supervisor. For instance, GSR work hours spent on mentoring interns or at external conferences
must be pre-approved by the supervisor.\\

\noindent
Graduate student working roles are all nominally in-person.

\subsubsection{Undegraduate Students}

Undergraduate students have the primary responsibility
of satisfying their degree requirements. Their research
activities should be in support of their professional goals.
Students working before their Junior year should be gaining
skills to prepare for more involved research in their
Junior and Senior years. Juniors wanting to pursue a 
graduate degree should be focused on 
completing a well-organized research project ahead of their
graduate applications. Seniors should focus on building their
professional skill sets for roles in industry or preparing
research skill sets for graduate school. \\

\noindent
Performing research and training to perform research are both
forms of labor and deserve compensation.  This labor could be
in the context of an academic course for credit (the compensation
is course credit) or as part of separate independent research (compensation
is pay). If Seniors are working
on a thesis project, they should be enrolled in the
related thesis research course. For undergraduates who
want to perform research for pay, the 
hiring process is handled through \href{https://careers.ucsc.edu/student/handshake-student-resources/index.html}{UCSC's HandShake system}. Positions are typically limited
to at most part-time 50\% FTE and are expected to be in-person,
including during summer quarter, unless otherwise arranged in advance
with Brant with written confirmation.




