\section{Logistics and Work Schedule}
\label{sec:logistics}
%when and for how long people work; policy on remote work; vacation; where the lab is located and how to get there (especially how to tell *others* how to get there); how to book a conference room; what to do if you get locked out

We have provided below some information on the logistics and work schedule. As noted throughout this document, for Postdoctoral Researchers and Graduate Student Researchers, the labor contracts dictate many of these policies and we refer to the contracts to provide the details. We are providing this information in goodwill to help
clarify expectations, rights, and responsibilities. Nothing here should be interpreted to supersede or replace
policies covered by labor agreements.


\subsubsection{Work Schedule and Location}

To provide the clearest possible information on work schedule and location expectations, here is a list:

\begin{itemize}
	\item Postdoctoral researchers and GSRs in the Comp Astro research group are expected to work in-person at their provided office workspace.
	\item Full time (100\% FTE) positions are expected to work during the nominal work hours listed in Section \ref{sec:nominal_work_hours}.
	\item Part time (50\% FTE) postions are expected to schedule their in-person work hours during the nominal work hours listed in Section \ref{sec:nominal_work_hours} and notify their supervisor about their schedule.
	\item Work schedules should accommodate mandatory meetings.
\end{itemize}

\noindent
The work schedule and mode for everyone in the group is subject to specific, pre-approved, and documented arrangements for modifications. For GSRs, time and effort commitment is covered by the \href{https://qa.ucnet.universityofcalifornia.edu/labor/bargaining-units/br/docs/br_28_time-and-effort-commitment_2022-2025.pdf}{UC GSR contract Article 28}. For Postdocs, time
and effort commitment is covered by the \href{https://ucnet.universityofcalifornia.edu/wp-content/uploads/labor/bargaining-units/px/docs/px_tentative_agremeents_effective_12-09-2022.pdf}{UC Postdoc contract Article 25}.\\

\noindent
When on authorized travel, the work schedule, location, and hours can vary. Travel for Postdoctoral Researchers is covered in \href{https://ucnet.universityofcalifornia.edu/wp-content/uploads/labor/bargaining-units/px/docs/px_tentative_agremeents_effective_12-09-2022.pdf}{UC Postdoc contract Article 28}. For GSRs, for the travel policy see \href{https://qa.ucnet.universityofcalifornia.edu/labor/bargaining-units/br/docs/br_29_travel_2022-2025.pdf}{UC GSR contract Article 29}.\\

\noindent
Disability accommodations may affect the nominal work schedule and location.

\subsubsection{Nominal Work Hours}
\label{sec:nominal_work_hours}

Our group's nominal work hours are 9am-5pm Pacific Time. In cases where external collaboration meetings require work during the 8am hour, which is possible given collaborations with researchers in Europe, nominal work hours may be shifted to 8am-4pm Pacific Time.

\subsubsection{Mandatory Group Meeting}

As part of the nominal work schedule, members of the CARG are required to attend the Comp Astro Research Group meeting. These mandatory meetings will
only be held during Nominal Work Hours (Section \ref{sec:nominal_work_hours}).

\subsubsection{Personal Time Off}

Both Postdoctoral researchers and GSRs have the ability to take personal time off (PTO) as listed in their contracts.\\

\noindent
For GSRs, PTO is
detailed in \href{https://ucnet.universityofcalifornia.edu/wp-content/uploads/labor/bargaining-units/br/docs/br_23_personal-time-off_2022-2025.pdf}{Article 23 of the UC GSR contract}, and is nominally up to 12 days per twelve-month period, and nominally requires advance approval. See the contract for details.\\


\noindent
For Postdocts, PTO is detailed in \href{https://ucnet.universityofcalifornia.edu/wp-content/uploads/labor/bargaining-units/px/docs/px_tentative_agremeents_effective_12-09-2022.pdf}{Article 17 of the UC Postdoc contract}, and is nominally up to 24 days per twelve-month period, and nominally requires advance approval. See the contract for details.


\subsubsection{Sick Leave}

Postdoctoral scholars are eligible for sick leave, as listed in \href{https://ucnet.universityofcalifornia.edu/wp-content/uploads/labor/bargaining-units/px/docs/px_tentative_agremeents_effective_12-09-2022.pdf}{UC Postdoc contract Article 23}. They nominally receive 12 work days per twelve-month appointment, all of which are available on the first day of the appointment.  See the contract for details.

\subsubsection{Short Term Leave}

GSRs are eligbile for short-term leave, as listed in \href{https://ucnet.universityofcalifornia.edu/wp-content/uploads/labor/bargaining-units/br/docs/br_17_leaves_2022-2025.pdf}{UC GSR contract Article 17}, including for personal illness or disability. The nominal eligibility is 2 days per quarter for a 50\% appointment and pro-rated for other appointments. See the contract for details.


\subsubsection{Other Forms of Leave and Absences}

Please refer to the Postdoctoral Researcher or GSR contracts for other forms of possible leave. 

\subsubsection{University Holidays and Calendar}

We observe the normal work calendar of the University of California, Santa Cruz, including its work holidays.
The UCSC official calendar can be found here:\\

\noindent
\href{https://registrar.ucsc.edu/calendar/academiccalendar.html}{UCSC Academic and Administrative Calendar}\\

\noindent
The university holidays encoded in the GSR contract are listed in \href{https://qa.ucnet.universityofcalifornia.edu/labor/bargaining-units/br/docs/br_14_holidays_2022-2025.pdf}{its Article 14}. For Postdocs, the university holidays are encoded in the \href{https://ucnet.universityofcalifornia.edu/wp-content/uploads/labor/bargaining-units/px/docs/px_tentative_agremeents_effective_12-09-2022.pdf}{Postdoc contract Article 8}.
