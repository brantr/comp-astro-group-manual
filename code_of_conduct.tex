\section{Code of Conduct and Resources}
\label{sec:code_of_conduct}

\subsection{Code of Conduct}

\subsubsection{Professionalism}

Performing forefront research requires professionalism. Our Code of Conduct requires professional interactions and that we satisfy expectations of professionalism. These expectations are detailed in this Manual, including especially Sections \ref{sec:roles}, \ref{sec:logistics}, \ref{sec:communication}, and \ref{sec:events}. Meeting these expectations
are requirements for continued Graduate Student Researcher (GSR) positions within the Comp Astro Research Group
and a PhD or Senior undergraduate thesis advising agreement with Brant Robertson.\\

\noindent
We note here a few baseline requirements:

\begin{itemize}
	\item Attendance at work and mandatory meetings: we expect people to be present at work during scheduled work hours and mandatory group meetings. See Sections \ref{sec:roles} and \ref{sec:events}.
	\item In-person work and meetings: we expect in-person work, in-person one-on-one meetings, and in-person group meetings. See Section \ref{sec:roles}, \ref{sec:logistics}, and \ref{sec:events} for more information.
	\item Prompt communication -- typical email response time is 24 hours during business days, see Section \ref{sec:communication}
\end{itemize}

\noindent
For graduate student researchers, the terms of employment are detailed in the \href{https://ucnet.universityofcalifornia.edu/resources/employment-policies-contracts/bargaining-units/graduate-student-researchers/contract/}{UC Graduate Student Researchers contract} and the
process for their discipline and dismissal is outlined in its \href{https://qa.ucnet.universityofcalifornia.edu/labor/bargaining-units/br/docs/br_07_discipline-and-dismissal_20225-2025.pdf}{Article 7 Discipline and Dismissal}.\\

\noindent
For postdoctoral researchers, the terms of employment are detailed in the
\href{https://ucnet.universityofcalifornia.edu/resources/employment-policies-contracts/bargaining-units/postdoctoral-scholars/contract/}{UC Postdoctoral Scholars contract} and the process
for their discipline and dismission is outlined in its 
\href{https://ucnet.universityofcalifornia.edu/wp-content/uploads/labor/bargaining-units/px/docs/px_tentative_agremeents_effective_12-09-2022.pdf}{Article 5 Discipline and Dismissal}.

\subsubsection{Nondiscrimination Statement}

Supporting ethical research requires ensuring a safe intellectual
environment where scientists can pursue their work. The UCSC
Computational Research Group will actively support all scientists
and does not discriminate on the 
basis of race, color, national origin, religion, sex, gender identity,
pregnancy, disability, age, medical condition (cancer-related), ancestry,
marital status, citizenship, sexual orientation, or status as a Vietnam-era
veteran or special disabled veteran.\\

\noindent
The official University of California nondiscrimination statement can
be found \href{https://www.ucop.edu/operating-budget/fees-and-enrollments/policies-and-resources/nondiscrimination-statement.html}{here}. Portions of this statement have been 
adapted for ours.

\subsubsection{Statement on Harassment}

Racial and sexual harassment will not be tolerated by any members of
the CARG. Harassment can be reported to Brant or to the Department Chair.
We are mandated reporters, and any harassment reported to us will be
referred to the University administration.

\subsection{Resources}

\subsubsection{UCSC Title IX Resources}

The UCSC Office of Diversity, Equity, and Inclusion resources are available at
\href{https://diversity.ucsc.edu/eeo-aa/eeo/speak_to_someone.html}{https://diversity.ucsc.edu/eeo-aa/eeo/speak\_to\_someone.html}.

\subsubsection{Student Grievances}
\label{sec:student_grievances}

\noindent
Information on filing a complaint or grievance with the UCSC Dean of Students Office
can be found at
\href{https://ada.ucsc.edu/about/grievance.html}{https://ada.ucsc.edu/about/grievance.html}

\subsubsection{Student-Related Nondiscrimination Contact}
Inquiries regarding the University's student-related nondiscrimination policies may be directed to Eric Heng, Student Affairs Immediate Office at (510) 987-0239.

\subsubsection{Counseling \& Psychological Services (CAPS)}

If you are encountering mental health challenges that impact your work, we want you to get access to resources that can help. We care about your well being, but within the Comp Astro Research Group we are not able to diagnose mental health issues or provide psychological counseling. However, UCSC does provide Counseling \& Psychological Services (CAPS).\\

\noindent
Students can either call or walk in to CAPS during business hours to schedule a first appointment, typically within 7 business days. The CAPS phone number is (831) 459-2628. When you call, you will be scheduled for a 30- to 40-minute first appointment with a CAPS counselor. During this appointment, the counselor will learn about your concerns and help you connect with resources and services. CAPS services are confidential, and most services are free. If you want more information before calling, you can visit the CAPS website: \\

\noindent
\href{https://caps.ucsc.edu/}{https://caps.ucsc.edu/}\\

\noindent
You may also contact the CAPS Let’s Talk drop-in consultation service to get brief information, advice, or feedback from a professional counselor. Let’s Talk is held several times a week at various locations around campus. You can read more about Let’s Talk on their website: \\

\noindent
\href{https://caps.ucsc.edu/resources/lets-talk/index.html}{https://caps.ucsc.edu/resources/lets-talk/index.html}\\

\subsubsection{Disability Resource Center (DRC)}

The Comp Astro Research Group is committed to creating an academic environment that supports its diverse
student body.  The \href{UCSC Disability Resource Center (DRC)}{https://drc.ucsc.edu/} provides resources for students with disabilities in support of their academic goals. The DRC can mandate accommodations for students, including extending the nominal time to degree. 
We encourage all students who may benefit from learning more about DRC services to contact the DRC by
phone at 831-459-2089 or by email at \href{mailto:drc@ucsc.edu}{drc@ucsc.edu}.

Regarding accommodations for GSR workers, please see \href{https://ucnet.universityofcalifornia.edu/wp-content/uploads/labor/bargaining-units/br/docs/br_24_reasonable-accommodation_2022-2025.pdf}{Article 24 on Reasonable Accommodations} of the UC GSR contract.
