\documentclass[fleqn,10pt]{wlscirep}
\usepackage[utf8]{inputenc}
\usepackage[T1]{fontenc}
\usepackage{bm}

%\documentclass[letter,11pt]{article}
\pdfoutput=1 % if your are submitting a pdflatex (i.e. if you have
             % images in pdf, png or jpg format)

\usepackage{jcappub} % for details on the use of the package, please
                     % see the JCAP-author-manual

\usepackage[T1]{fontenc}

\title{\boldmath UCSC Computational Astrophysics Research Group Manual}


%% %simple case: 2 authors, same institution
%% \author{A. Uthor}
%% \author{and A. Nother Author}
%% \affiliation{Institution,\\Address, Country}

% more complex case: 4 authors, 3 institutions, 2 footnotes

\author[1]{Brant Robertson}
\affiliation[1]{Department of Astronomy and Astrophysics, University of California, Santa Cruz, Santa Cruz, CA 95064}

% e-mail addresses: one for each author, in the same order as the authors
\emailAdd{brant@ucsc.edu}



\abstract{\\\\
The University of California, Santa Cruz Computational Astrophysics Research Group consists of students, postdoctoral scholars, and faculty conducting
forefront astrophysics research using computationally-intensive means. Our research efforts
include hydrodynamical and cosmological simulation,
deep learning applied to astronomy, and analysis of
large volume data. This Research Group Manual provides an overview of our policies including our Code of Conduct, and information on resources, mentorship,
and best practices.
}



\begin{document}
\maketitle
\flushbottom

%%%%% MISSION STATEMENT
\newpage
\section{Mission Statement}
\label{sec:mission}
{\bf UCSC Computational Astrophysics Research Group} is a collaboration of
research scholars who leverage state of the art numerical methods to conduct
forefront science.


%%%%%%%%% OVERVIEW
\newpage
\section{Overview}
\label{sec:overview}
UCSC Computational Astrophysics Research Group (CARG).


%%%%% CODE OF CONDUCT
\newpage
\section{Code of Conduct and Resources}
\label{sec:code_of_conduct}

\subsection{Code of Conduct}

\subsubsection{Professionalism}

Performing forefront research requires professionalism. Our Code of Conduct requires professional interactions and that we satisfy expectations of professionalism. These expectations are detailed in this Manual, including especially Sections \ref{sec:roles}, \ref{sec:logistics}, \ref{sec:communication}, and \ref{sec:events}. Meeting these expectations
are requirements for continued Graduate Student Researcher (GSR) positions within the Comp Astro Research Group
and a PhD or Senior undergraduate thesis advising agreement with Brant Robertson.\\

\noindent
We note here a few baseline requirements:

\begin{itemize}
	\item Attendance at work and mandatory meetings: we expect people to be present at work during scheduled work hours and mandatory group meetings. See Sections \ref{sec:roles} and \ref{sec:events}.
	\item In-person work and meetings: we expect in-person work, in-person one-on-one meetings, and in-person group meetings. See Section \ref{sec:roles}, \ref{sec:logistics}, and \ref{sec:events} for more information.
	\item Prompt communication -- typical email response time is 24 hours during business days, see Section \ref{sec:communication}
\end{itemize}

\noindent
For graduate student researchers, the terms of employment are detailed in the \href{https://ucnet.universityofcalifornia.edu/resources/employment-policies-contracts/bargaining-units/graduate-student-researchers/contract/}{UC Graduate Student Researchers contract} and the
process for their discipline and dismissal is outlined in its \href{https://qa.ucnet.universityofcalifornia.edu/labor/bargaining-units/br/docs/br_07_discipline-and-dismissal_20225-2025.pdf}{Article 7 Discipline and Dismissal}.\\

\noindent
For postdoctoral researchers, the terms of employment are detailed in the
\href{https://ucnet.universityofcalifornia.edu/resources/employment-policies-contracts/bargaining-units/postdoctoral-scholars/contract/}{UC Postdoctoral Scholars contract} and the process
for their discipline and dismission is outlined in its 
\href{https://ucnet.universityofcalifornia.edu/wp-content/uploads/labor/bargaining-units/px/docs/px_tentative_agremeents_effective_12-09-2022.pdf}{Article 5 Discipline and Dismissal}.

\subsubsection{Nondiscrimination Statement}

Supporting ethical research requires ensuring a safe intellectual
environment where scientists can pursue their work. The UCSC
Computational Research Group will actively support all scientists
and does not discriminate on the 
basis of race, color, national origin, religion, sex, gender identity,
pregnancy, disability, age, medical condition (cancer-related), ancestry,
marital status, citizenship, sexual orientation, or status as a Vietnam-era
veteran or special disabled veteran.\\

\noindent
The official University of California nondiscrimination statement can
be found \href{https://www.ucop.edu/operating-budget/fees-and-enrollments/policies-and-resources/nondiscrimination-statement.html}{here}. Portions of this statement have been 
adapted for ours.

\subsubsection{Statement on Harassment}

Racial and sexual harassment will not be tolerated by any members of
the CARG. Harassment can be reported to Brant or to the Department Chair.
We are mandated reporters, and any harassment reported to us will be
referred to the University administration.

\subsection{Resources}

\subsubsection{UCSC Title IX Resources}

The UCSC Office of Diversity, Equity, and Inclusion resources are available at
\href{https://diversity.ucsc.edu/eeo-aa/eeo/speak_to_someone.html}{https://diversity.ucsc.edu/eeo-aa/eeo/speak\_to\_someone.html}.

\subsubsection{Student Grievances}
\label{sec:student_grievances}

\noindent
Information on filing a complaint or grievance with the UCSC Dean of Students Office
can be found at
\href{https://ada.ucsc.edu/about/grievance.html}{https://ada.ucsc.edu/about/grievance.html}

\subsubsection{Student-Related Nondiscrimination Contact}
Inquiries regarding the University's student-related nondiscrimination policies may be directed to Eric Heng, Student Affairs Immediate Office at (510) 987-0239.

\subsubsection{Counseling \& Psychological Services (CAPS)}

If you are encountering mental health challenges that impact your work, we want you to get access to resources that can help. We care about your well being, but within the Comp Astro Research Group we are not able to diagnose mental health issues or provide psychological counseling. However, UCSC does provide Counseling \& Psychological Services (CAPS).\\

\noindent
Students can either call or walk in to CAPS during business hours to schedule a first appointment, typically within 7 business days. The CAPS phone number is (831) 459-2628. When you call, you will be scheduled for a 30- to 40-minute first appointment with a CAPS counselor. During this appointment, the counselor will learn about your concerns and help you connect with resources and services. CAPS services are confidential, and most services are free. If you want more information before calling, you can visit the CAPS website: \\

\noindent
\href{https://caps.ucsc.edu/}{https://caps.ucsc.edu/}\\

\noindent
You may also contact the CAPS Let’s Talk drop-in consultation service to get brief information, advice, or feedback from a professional counselor. Let’s Talk is held several times a week at various locations around campus. You can read more about Let’s Talk on their website: \\

\noindent
\href{https://caps.ucsc.edu/resources/lets-talk/index.html}{https://caps.ucsc.edu/resources/lets-talk/index.html}\\

\subsubsection{Disability Resource Center (DRC)}

The Comp Astro Research Group is committed to creating an academic environment that supports its diverse
student body.  The \href{UCSC Disability Resource Center (DRC)}{https://drc.ucsc.edu/} provides resources for students with disabilities in support of their academic goals. The DRC can mandate accommodations for students, including extending the nominal time to degree. 
We encourage all students who may benefit from learning more about DRC services to contact the DRC by
phone at 831-459-2089 or by email at \href{mailto:drc@ucsc.edu}{drc@ucsc.edu}.

Regarding accommodations for GSR workers, please see \href{https://ucnet.universityofcalifornia.edu/wp-content/uploads/labor/bargaining-units/br/docs/br_24_reasonable-accommodation_2022-2025.pdf}{Article 24 on Reasonable Accommodations} of the UC GSR contract.



\newpage
\section{Roles \& Expectations}
\label{sec:roles}

This section outlines the typical work responsibilities members of the Comp Astro Research Group.
For everyone in the CARG,
the typical work schedule and mode is described in Section \ref{sec:logistics}.
The terms of employment are provided in the university offer letters, which typically come from the Dean's hiring authority.

\subsubsection{Faculty}

Brant's primary responsiblity
is to enable the research
of other CARG members. He
must keep the lights on,
find resources, edit
papers, and serve on departmental
and professional committees. These
responsibilities are in addition to his
teaching role.\\

\noindent
He also will occasionally try to
write his own papers!


\subsubsection{Postdoctoral Scholars}

Postdocs in the CARG have
a primary responsibility
to pursue their research 
projects. A postdoctoral research position is a nominal 40-hour per week 
in-person time commitment for a 100\% FTE.

\subsubsection{Graduate Students}

Graduate students in the CARG have
different responsibilities
depending on whether they
are on TA, GSR, or summer GSR, and
their career stage.

\begin{itemize}
	\item For students being supported on a TA, the role is entirely academic and focused on making nominal progress toward the PhD degree (including thesis research).
	\item For students being supported on a GSR during the academic year, the role is a combination of academic efforts making nominal progress toward the PhD and conducting directed research. The directed research portion funded by the GSR during the academic year is a nominal 20-hour per week time commitment for a 50\% FTE.
	\item For students on summer GSR, the role is completely directed research funded by the GSR. The
	directed research portion funded by the GSR during the summer quarter is a nominal 40-hour per week time commitment for a 100\% FTE.
\end{itemize}

\noindent
When working on a paid GSR, work activites that count toward the employment are pre-approved by the supervisor. For instance, GSR work hours spent on mentoring interns or at external conferences
must be pre-approved by the supervisor.\\

\noindent
Graduate student working roles are all nominally in-person.

\subsubsection{Undegraduate Students}

Undergraduate students have the primary responsibility
of satisfying their degree requirements. Their research
activities should be in support of their professional goals.
Students working before their Junior year should be gaining
skills to prepare for more involved research in their
Junior and Senior years. Juniors wanting to pursue a 
graduate degree should be focused on 
completing a well-organized research project ahead of their
graduate applications. Seniors should focus on building their
professional skill sets for roles in industry or preparing
research skill sets for graduate school. \\

\noindent
Performing research and training to perform research are both
forms of labor and deserve compensation.  This labor could be
in the context of an academic course for credit (the compensation
is course credit) or as part of separate independent research (compensation
is pay). If Seniors are working
on a thesis project, they should be enrolled in the
related thesis research course. For undergraduates who
want to perform research for pay, the 
hiring process is handled through \href{https://careers.ucsc.edu/student/handshake-student-resources/index.html}{UCSC's HandShake system}. Positions are typically limited
to at most part-time 50\% FTE and are expected to be in-person,
including during summer quarter, unless otherwise arranged in advance
with Brant with written confirmation.






%%% OPEN SCIENCE
%\newpage
%\section{Open Science}
%\label{sec:open_science}
%why and how the lab deals with sharing data, code, materials; how the lab does version control; how to do this stuff with lab collaborators 

%%% COMMUNICATIONS
\newpage
\section{Communication}
\label{sec:communication}
%how people in the lab talk to each other; things like "always use Slack", "phone calls are only for emergencies"; how long to wait before nudging the PI on something you need; how meetings with PI work and what to prepare for them; check-ins/stand-ups/huddles


\subsubsection{Slack}
Slack is the primary mode
of communication for the CARG,
through \href{http://ucsc-comp-astro.slack.com}{ucsc-comp-astro.slack.com}.
Ask Brant for an invitation.

\subsubsection{Email}
We maintain a group email
list \href{mailto:comp-astro-group@ucsc.edu}{comp-astro-group@ucsc.edu}.
We will use this email list for
announcements. Your UCSC email
will be signed up for a variety
of other lists managed by the
university and department.

\subsubsection{Response Timeline}
Expectations for CARG member
communications on a reasonable timescale, typically within
24 hours for email and as
permitting via Slack.
We do not expect communications
from CARG members after business
hours, over the weekend, or
on holidays. If you receive
communications from Brant or
another CARG member outside of
business hours, please assume
it is because
they had the opportunity
to do so and not because they
expect you to communicate during
those times.

\subsubsection{Websites}
Please develop and maintain a
professional website. A good
option is a GitHub.io site (for
instance, Brant's \href{http://brantr.github.io}{github.io site}).
There is a 
\href{http://robertson.sites.ucsc.edu}{UCSC Computational Astrophysics Research Group website}. We can
cross-link sites as desired.





%%% LOGISTICS
%\newpage
%\section{Logistics}
%\label{sec:logistics}
%when and for how long people work; policy on remote work; vacation; where the lab is located and how to get there (especially how to tell *others* how to get there); how to book a conference room; what to do if you get locked out

%%% INTERNAL RESOURCES
%\newpage
%\section{Internal Resources}
%\label{sec:resources}
%these are usually a long list of things people in the lab need access to: servers, software packages, commonly used web tools, shared credentials, room keys and other physical resources, etc


%%% FACILITIES
%\newpage
%\section{Facilities}
%\label{sec:facilities}
%how to get a library card; which building has the best photocopier; useful websites and tutorials, etc

%%%% ONBOARDING
%\newpage
%\section{Onboarding}
%\label{sec:onboarding}
%master list of everything that new lab members need to deal with when starting out, including credentials, software, hardware, keys, university ID, etc


%%% EVENTS
\newpage
\section{Events}
\label{sec:events}

\subsubsection{Group Meetings}
\label{sec:group_meetings}
Mandatory in-person group meetings will be held, usually
weekly, and will be announced a week in advance.
A typical location is the CfAO Atrium interaction
area, typically on Mondays at 11am or Tuesdays at 10am.
We discuss current events,
research activities, university
happenings, etc. Group members
may be asked to share their
current successes and challenges,
or present slides with advanced
notice.
The in-person group meetings are mandatory
for CARG members.

\subsubsection{Cosmo Club}
\label{sec:cosmo_club}

The cosmology and galaxy
formation-focused weekly
seminar is Cosmo Club, 
which is held
Mondays 12:30pm - 1:30pm in
ISB 102. CARG members should
attend this weekly seminar
and sign up 
to meet with speakers.

\subsubsection{Astronomy Colloquia}
\label{sec:astro_colloquia}

The Astronomy colloquia are a
critical event for the
intellectual culture of the
Astronomy department, and all
members of CARG should attend
every week as possible.
They are held on Wednesdays
at 10 AM - 12PM, including pre-colloquium Tea
(usually in CfAO) and the (usually in
Nat Sci Annex 101).\\

\noindent
Usually, a meeting schedule is 
created for the colloquium speaker each
week. Please sign up to meet
with every colloquium speaker as
schedule permits.\\

\noindent
The
\href{https://www.astro.ucsc.edu/news-events/Seminars/index.html}{Astronomy Colloquium Schedule is available here}.

\subsubsection{Physics Colloquia}
\label{sec:phys_colloquia}

The Physics Department Colloquia are usually
scheduled from 3:45 pm-4:55 pm on Thursdays
in Physical Sciences 114. They serve
cookies at 3:20pm in the ISB foyer. 
CARG members are encourage to attend
this optional event.\\

\noindent
The
\href{https://www.physics.ucsc.edu/news-events/colloquia/index.html}{Physics Colloquium Schedule is available here}.

\subsubsection{FLASH}
\label{sec:flash}

The weekly informal departmental
seminar is the
Friday Lunch Time Astrophysics Seminar,
which is held
Fridays 12pm - 1pm, usually in ISB 102.
CARG members should attend this
weekly event.\\

\noindent
In the Spring, second year 
graduate students
usually present their research at
FLASH. It's especially important
to attend FLASH during this
time to support our students.



%%% TEAM CONSIDERATIONS
%\newpage
%\section{Team Considerations}
%\label{sec:team}
%dress code \& hygiene; can you bring your dog to the lab; person X has a peanut allergy so please pack your lunch carefully; health stuff; work-life balance

%%% ETHICS
%\newpage
%\section{Ethics}
%\label{sec:ethics}
%irb procedures; safety procedures; what to do when something goes wrong; what to do/who to call in an emergency


%%%% ENGAGEMENT: social media, press, websites
\newpage
\section{Engagement}
\label{sec:engagement}

Public outreach,
social media, and 
press engagement can
all form portions of our
work. You are of course 
welcome to pursue these
activities on your own
time and these will invariably
mix with professional 
activities. Below we 
list some related
guidelines.


\subsubsection{Social Media}
\label{sec:social_media}

Social media consist
of an important
platforms for communication
and networking.
When representing CARG
on social media, through
communicating your results,
discussing conference
activites, etc., please 
aim to be effective and
professional.
Ultimately, we are responsible
for maintaining our own
professional reputations.\\
\noindent
We do not currently maintain
group social media accounts.
Brant has a Twitter 
account, but he's been unable to access
it for a few years!
Other social networks
include Facebook and
LinkedIn.

\subsubsection{Press Engagement}
\label{sec:press_engagement}

Regarding press activity,
any engagement about
CARG research should be
coordinated with Brant.
He commits to include
and center other CARG 
members as appropriate.

\subsubsection{Public Outreach}
\label{sec:public_outreach}

Public outreach is an
important facet of our
mission as scientists.
Treat public outreach
as professionally as
your research. The
CARG will share
outreach resources.
Please responsibly manage
the relative prioritization
of
research and outreach.



%%% RESEARCH
%\newpage
%\section{Research}
%\label{sec:research}
%detailed instructions for doing research; some labs do this for all experiments separately while others have a general set of instructions like "how you should interact with participants"; how to compensate participants

%\subsubsection{Code Development}
%\label{sec:code}

%how the lab does analyses; expectations concerning how code is written, version controlled, and archived; how to handle data protection and security. this one overlaps with OPEN SCIENCE quite a bit


%% PUBLICATIONS
%\newpage
%\section{Publications}
%\label{sec:publications}
%how the lab deals with authorship, including the differences between listing in acknowledgments \& co-authors; checklist of everything to do before a paper is published; policies on preprints, postprints, and open access; preferences about journals

%%% CONFERENCES
%\newpage
%\section{Conferences}
%\label{sec:conferences}
%how to give a talk; which conferences do lab members usually attend and why; technology considerations surrounding visualizations; data considerations for work-in-progress; discussion of \#betterposter or \#worseposter, etc


%%%%% FINANCES
\newpage
\section{Finances \& Reimbursements}
\label{sec:finances}

The CARG is funded primarily by
NASA,
the National Science Foundation,
and Hubble and James Webb Space
Telescope-related grants. 

\subsubsection{Accounts}
For reference, here is a listing of all the CARG accounts and FOAPALs.
\begin{itemize}
\item NSF MRI 81428-443162
\item NASA ATP 83290-443162
\item JWST NIRCam 63513-443162
\item WFIRST 83263-443162
\item LACES 58437-443162 (defunct)
\end{itemize}

\subsubsection{Travel Reimbursements}

Always get pre-approval from 
Brant on travel,
including conference fees,
flight arrangements, 
hotel arrangements, etc. 
University policies related to
travel can be found at \href{https://financial.ucsc.edu/Pages/Travel_Process.aspx}{https://financial.ucsc.edu/Pages/Travel\_Process.aspx}. These resources
supersede any of Brant's 
understandings of reimbursement
policies.\\

\noindent
Perhaps the most important form is the 
\href{https://financial.ucsc.edu/Financial_Affairs_Forms/Post_Travel_Expense.pdf}{Post Travel Form}, which
you will need to request reimbursement.
This form should be filled out by
the CARG members who are 
requesting the reimbursement, and
then provided to Brant to fill in
the FOAPAL info and authorization.
If you need help, ask Brant.


%%% OFFBOARDING
%\newpage
%\section{Offboarding}
%\label{sec:offboarding}
%what to do when you leave the lab; making sure all data \& materials are archived properly; transferring credentials; how to stay in touch (what happens to your Slack account)

%MENTORSHIP AND DEVELOPMENT
%\newpage
%\section{Mentorship \& Development}
%\label{sec:mentorship}
%how to choose a project; how to get feedback from others on new ideas; how to initiate collaborations inside/outside the lab; doing a thesis/dissertation; professional development at university level; rec letters


%%% WHERE TO GET HEP
%\section{Where to Get Help}
%\label{sec:help}
%how to get support from inside or outside the lab; schedule of routine training for new members; things that bear repeating like "always ask questions!"; how to make mistakes productively


%%% READING LIST
%\newpage
%\section{Reading List}
%\label{sec:reading_list}

\newpage
\section{Sample Mentoring Agreement}

TAKEN FROM CITL

This document provides an adaptable structure and set of topics to discuss when setting up a mentorship relationship. Both mentors and mentees can review their responses to these questions and work together to establish guidelines for their mentorship relationship, toward the end of supporting the mentee to succeed in their research program and to make progress toward their personal and professional goals. This document can be developed collaboratively and revisited whenever necessary.


1. Shared Goals \& Vision (List the goals of this working relationship. What do you both hope to get out of working together? What skills and experience will the mentee gain through the project, and how will that learning serve their larger academic, personal, professional goals?)

\begin{itemize}	
\item Research project goals:
\item Mentee’s personal/professional goals:
\item Mentor’s personal/professional goals: 
\item Shared vision of success for this project: 
\end{itemize}

2.	Approaches/Strategies/Steps to achieving listed goals (What do you both need to do to meet the above goals? Who is responsible for what actions?)
 
\begin{itemize}
\item Mentor’s role/tasks:
\item Mentee’s role/tasks:
\end{itemize}
 

3.	Meeting Practices (Frequency, duration, format and platform for meetings, who will schedule the meetings, etc.)



4.	Meeting Preparation (What will the mentee do to prepare for meetings? What will the mentor do to prepare for meetings?)
 
 
 
5.	Communication Etiquette (How will the mentor and mentee communicate between meetings? Establish preferred modes of contact, timeframe for responsiveness, when to contact faculty mentor v. when to contact graduate student mentor, etc.)



6.	Unplanned Issues (How will the mentor and mentee address unplanned issues that come up? For example, if the mentee gets stuck while working on the project, what are some steps they can take?)



7.	Confidentiality Concerns (Ensure you are on the same page in terms of keeping particular discussions in confidence. If there are topics which either of you feel are off-limits for discussion, those can be named.)
 





\newpage
\section{Acknowledgments}

This group manual was inspired by this \href{https://twitter.com/samuelmehr/status/1139733291899080705}{tweet from @samuelmehr}. We have also made use of some information provided by the \href{https://citl.ucsc.edu}{UCSC Center for Innovations in Teaching and Learning}.

% The bibliography will probably be heavily edited during typesetting.
% We'll parse it and, using the arxiv number or the journal data, will
% query inspire, trying to verify the data (this will probalby spot
% eventual typos) and retrive the document DOI and eventual errata.
% We however suggest to always provide author, title and journal data:
% in short all the informations that clearly identify a document.

%\begin{thebibliography}{99}
%\bibitem{a}
%Author, \emph{Title}, \emph{J. %Abbrev.} {\bf vol} (year) pg.
%\bibitem{b}
%Author, \emph{Title},
%arxiv:1234.5678.

%\bibitem{c}
%Author, \emph{Title},
%Publisher (year).


% Please avoid comments such as "For a review'', "For some examples",
% "and references therein" or move them in the text. In general,
% please leave only references in the bibliography and move all
% accessory text in footnotes.

% Also, please have only one work for each \bibitem.
%\end{thebibliography}
\end{document}
